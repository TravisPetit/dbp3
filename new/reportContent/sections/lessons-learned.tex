\section{Lessons Learned}
The main lesson we learned during this project is not to underestimate how time consuming working with data is.
Both during the integration and the analysis phase, we were surprised by how long certain operations took and had to cut some corners.
The take away is not only to start earlier, but also to use better tools and put more thought into optimization up front when dealing with large datasets.
During the integration phase, we used python and looped through every single record individually and inserted them into the database one by one.
We believed that this would be the easiest way for us to go about it and that it would be sufficiently efficient since we still had quite a few days left before the deadline.
As it turned out, we very nearly didn't get done in time. If we had taken the time to use a more performant language and had put the work in to send batch updates to the SQL server, we likely could have cut down our execution time by a few orders of magnitude.

The problem we encountered during the analysis phase, which is laid out in the Methods section, showed us again how vital efficiency is when working with data.
We wasted a considerable amount of time waiting for a needlessly inefficient query to finish before realizing that adding an index structure and rewriting the query slightly would make it orders of magnitude faster.
On the other hand, our issues during the integration phase did make us more open to the idea that optimization is important and caused us to investigate relatively quickly.
As a result we managed to get the desired chart ready in time for the deadline.

%Moving on, in terms of the data visalization, probably the most interesting fact we have seen is the margin by which the number of
%suicides have increased in the United States in the year span from 1999 to 2017 as showin in \ref{fig:yearly_suicides}.

%Other charts do show some connections between the results of the elections in 2016 and the suicide rates in the years 2016 and
%2017. In short, there is small 

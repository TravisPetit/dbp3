\section{Source material}

The choice of sources is paramount when it comes to doing projects
such as this one. Our datasets are:
\begin{enumerate}
	\item \textbf{2012 and 2016 US Presidential Elections} \cite{kaggle-website}
		This dataset contains the result of the last two US presidential
		election per county.

	\item \textbf{The GDELT2 events dataset} \cite{gdelt-website}
		GDELT, which stands for Global Database of Events, Language
		and Tone is a database which monitors and manages
		political episodes gobally. It is one of the largest
		open souce indexes of global society
		and it keeps track of ocurrances
		such as diplomatic summits, wars, news, protests et
		cetera as well as the relationships between them.


		The GDELT 2 dataset consists of three main parts: an
		event dataset, a mentions dataset and a global knowledge
		graph. For this project work with the events dataset
		which is itself divided into one hundred and sixty three
		thousand csv files separated by timestamps which cover
		global events ranging from the year 2015 onwards.
		Because of the specifications of the project, we decided
		to work with the eighty nine thousand of them, namely
		those in the interval starting from January of 2015 to
		September of 2017, nine months after Donald Trump
		was declared president of the United States.


	\item \textbf{Centers for disease and control prevention's
		suicide rate per US county reports}\cite{suicide-website}
		The Centers for disease and control website have an online
		tool that lets users download customized datasets.
		We opted for a csv file which contains information only
		on suicides.

	\item \textbf{ERSI's shapefile of US counties}\cite{esri-website}
		This dataset has been added during integration to
		match counties from diferent datasets together
		using geo-locations
		as we could not do that using
		their FIPS codes because each dataset's FIPS codes
		were contradictory to our other datasets.
		The shapefile was converted to sql and can be found
		on /sql\_scripts/uscounties.sql on our repository.


\end{enumerate}
All of the sources combined amount to about 85 GB of data.

\textbf{Note}: For the P1 milestone we presented a slightly
different version of sources: we sated that we were going to use
the GDELT1 (as opposed to the GDELT2) events dataset.
This was because we felt that since both of thsese sources have
more data than we need, going for GDELT2 would be an overkill,
however we realized later on that unlike for GDELT1,
the GDELT2 csv files contain a record called ADM2Code,
which can be used to link counties and events together.
This change rocketed our file sizes from 23 GB to more than
80 GB. Because of this, we decided to drop another source:
The Crowd Counting Consortium's protest activity dataset, since
the information found thre is redundant now that we have
GDELT2.

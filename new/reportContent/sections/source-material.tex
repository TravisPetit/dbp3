\section{Source material}

The choice of sources is paramount when it comes to doing projects
such as this one. Our datasets are:
\begin{enumerate}
	\item \textbf{A US county level presidatial resul for the
		years 2012 to 2016}
		{\color{red}(add citaton)}
		The reason for which we use the outcomes
		of the 2012 and 2016 elections instead of just the latter
		is that eventrs from 2012 onwards are used as a
		control group, i.e. we \ldots

	\item \textbf{The GDELT2 events dataset}
		GDELT, which stands for Global Database of Events, Language
		and Tone is a database which monitors and manages
		political episodes gobally. It is one of the largest
		open souce indexes of global society {\color{red} citaton}
		and it keeps track of ocurrances
		such as diplomatic summits, wars, news, protests et
		cetera as well as the relationships between them.


		The GDELT 2 dataset is consists of three parts: an
		event dataset, a mentions dataset and a global knowledge
		graph. For this project work with the events dataset
		which is itself divided into one hundred and sixty three
		thousand csv files separated by timestamps which cover
		global events ranging from the year 2015 onwards.
		Because of the specifications of the project, we decided
		to work with the eighty nine thousand of them, namely
		those in the interval starting from January of 2015 to
		September of 2017, nine months after Donald Trump
		was declared president of the United States.

	\item \textbf{Centers for disease and control prevention's
		suicide rate per US county reports}

\end{enumerate}
All of the sources combined amount to about 85 GB of data.

\textbf{Note}: For the P1 milestone we presented a slightly
different version of sources: we sated that we were going to use
the GDELT1 (as opposed to the GDELT2) events dataset.
This was because we felt that since both of thsese sources have
more data than we need, going for GDELT2 would be an overkill,
however we realized later on that unlike for GDELT1,
the GDELT2 csv files contain a record called ADM2Code,
which can be used to link counties and events together
(which is called for, details on this are covered
on the INTEGRATION section).
This change rocketed our file sizes from 23 GiB to more than
80 GiB. Because of this, we decided to drop another source:
The Crowd Counting Consortium's protest activity dataset, since
the information found thre is redundant now that we have
GDELT2.

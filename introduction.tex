The rise of social media has given people from all different
backgrounds a plattform with which to share their thoughts
and feelings online leaving them avaiable for anyone to read.
This phenomenon gives society yet another tool for analysing
people's mass behavior; after all, if many individuals share
similar posts online, it may be a sign that something
is going on.
This becomes particualrly interesting
when important events that affect millons of people take
palce.

After the last US elections
some were hearly displeased and others were
satisfied, this friction has been pronounced by the fact
that the controvesial figure Donald Trump has taken over
the presiency of the United States since 2016. In fact many opposers of Trump did not stop at Twitter or Reddit, they
took it up to themselves to go out and protest on the streets.

One must not forget how much impact such events can have over
the mental well-being of people.
%But just how bad has the sate of affairs been ever since? This is not a trivial question.

Our goal with this project is to answer the question
"To what extent have the results of the US 2016 presidential
elections and its consequences
affected the suicide rates in the United States?"
To do so we use information about protests, riots, and
related events together with the outcomes of the elections
and reports regarding suicide rates all neatly stored in a database
to make a county-level investigation
regarding the question at hand.




%{\color{red} sources}\\

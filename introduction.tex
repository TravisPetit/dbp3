{\color{red}problem}\\
"To what extent have the result of the US 2016 presidential
elections affected the suicide rates in the United States?"

{\color{red} analysis goals}\\
The rise of social media has given people from all different
backgrounds a plattform with which to share their thoughts
and feelings online leaving them avaiable for anyone to read.
This phenomenon gives society yet another tool for analysing
people's mass behavior; after all, if many individuals share
similar posts online, it may be a sign that something
is going on.
This becomes particualrly interesting
when important events that affect everyone in a contry take
palce.

After the last US elections,
some were hearly displeased and others
satisfied, this friction has been pronounced by the fact
that the controvesial figure Donald Trump has taken over
the presiency of the United States since 2016.

Many opposers of Trump did not stop at Twitter or Reddit, they
took it up to themselves to go out and protest on the streets.

Our aim is to use information about protests, riots, and
related events together with the results of the elections
and the suicide rates to make a conty-level investigation
regarding the question above.



%{\color{red} sources}\\
